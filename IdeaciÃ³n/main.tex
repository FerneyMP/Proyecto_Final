\documentclass{article}
\usepackage[utf8]{inputenc}
\usepackage[spanish]{babel}
\usepackage{listings}
\usepackage{graphicx}
\graphicspath{ {images/} }
\usepackage{cite}

\begin{document}

\begin{titlepage}
    \begin{center}
        \vspace*{1cm}
            
        \Huge
        \textbf{Proyecto final}
            
        \vspace{0.5cm}
        \LARGE
        Informática II
            
            
        \vspace{1.5cm}
        
        \textbf{Marcela Flórez Orellano} 
        
        \vspace{0.3cm}
        \LARGE
        
        \textbf{Ferney Mejía Pérez}
            
        \vfill
            
        \vspace{0.8cm}
            
        \Large
        Departamento de Ingeniería Electrónica y Telecomunicaciones\\
        Universidad de Antioquia\\
        Medellín-Antioquia\\
        23 de Marzo de 2021
         
            
    \end{center}
\end{titlepage}


\tableofcontents
\newpage


\section{Sección introductoria}\label{intro}
El presente proyecto hace énfasis en la implementación de las ideas principales del proyecto final correspondiente al curso en proceso, informática II, la cuál tiene un enfoque de enseñanza en el lenguaje de programación C++, de tal motivo que, se realizarán diversos ejercicios de aprendizaje con el fin de adquirir lógica computacional y la habilidad necesaria para la solución de problemas comunes en la programación. Por lo tanto, para este proyecto se tiene de inicio la creación del esquema organizacional del cual se desprenderán las diferentes ideas que componen el cuerpo del juego correspondientes al mencionado proyecto.

Ahora bien, con respecto al juego, se tiene una idea global del juego, en la cual se puede competir en diferentes galaxias para ser el mejor piloto intergaláctico del universo conocible. Se cuenta con diversos mapas en los que los contenidos galácticos cambian, igual que los competidores. 
\section{Sección de contenido} \label{contenido}
A continuación se presentan las ideas pertenecientes al esquema organizacional del juego:

\subsection{Entorno}
El juego tendrá un entorno desarrollado en el espacio sideral, en el cuál habrán múltiples galaxias a las cuales acceder para realizar competiciones en un mínimo de tiempo determinado.
\subsection{Protagonistas}
Se contarán con un par de personajes que están en búsqueda de ser los mejores pilotos intergalácticos, los cuales dispondrán de varias naves con tecnología cuántica capaz de sobrepasar las leyes de la física.

\subsubsection{Antagonistas}
Criaturas desconocidas que aumentan su poder y capacidad de ataque a medida que se avanza de nivel. Estos monstruos aparecerán y desaparecerán de forma aleatoria a lo largo del juego, con el objetivo de retrasar la llegada de los jugadores a la meta.

\subsection{Acciones}
Los jugadores se enfrentarán a diferentes tipos de ataques por parte de los antagonistas, como lo son, disparos de asteroides de diversos tamaños, bombas de energía o polvo espacial.

\subsection{Recursos}
A lo largo de la pista de carrera se encontrarán cajas con diversos poderes que se le otorgaran al jugador, por ejemplo, la capacidad de acelerar a la velocidad de la luz o atravesar por un agujero de gusano, tratándose de un atajo para alcanzar la meta.

\subsection{Eliminación}
Los jugadores pierden puntos (o se termina el juego, por definir ) si no logran llegar a la meta en el tiempo estipulado o si llegan a ser atacados.

\subsection{Interacción}
Los jugadores podrán eliminar a las criaturas que irán emergiendo mientras se recorren las pistas galácticas.

\subsection{Modalidad}
El juego se diseñará para un solo jugador, quien podrá elegir distintas naves de carrera intergaláctica.

\subsection{Dificultad}
Mientras más avanza el jugador en las diferentes galaxias, más irá costando el ganar dentro de los diferentes mundos de acción, es decir, el nivel de dificultad aumentará pese tenga más recorrido y haya sobrepasado niveles inferiores. Además, las criaturas galácticas que aparecen en los niveles más altos, contarán con una tecnología más desarrollada, lo que implicará que esquivar sus ataques y/o destruirlos tomará más tiempo de lo habitual. 

\subsection{Finalidad}
Recolección de galaxias, planetas, estrellas, lo cual es necesario para convertirse en el mejor piloto intergaláctico y más experimentado de todo el universo conocible. 
\newpage
\section{Referencias}
\bibliographystyle{IEEEtran}
\bibliography{references}
\cite{calistenia}


\end{document}
